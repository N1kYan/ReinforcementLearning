%%%%%%%%%%%%%%%%%%%%%%% file template.tex %%%%%%%%%%%%%%%%%%%%%%%%%
%
% This is a general template file for the LaTeX package SVJour3
% for Springer journals.          Springer Heidelberg 2010/09/16
%
% Copy it to a new file with a new name and use it as the basis
% for your article. Delete % signs as needed.
%
% This template includes a few options for different layouts and
% content for various journals. Please consult a previous issue of
% your journal as needed.
%
%%%%%%%%%%%%%%%%%%%%%%%%%%%%%%%%%%%%%%%%%%%%%%%%%%%%%%%%%%%%%%%%%%%
%
% First comes an example EPS file -- just ignore it and
% proceed on the \documentclass line
% your LaTeX will extract the file if required
\begin{filecontents*}{example.eps}
%!PS-Adobe-3.0 EPSF-3.0
%%BoundingBox: 19 19 221 221
%%CreationDate: Mon Sep 29 1997
%%Creator: programmed by hand (JK)
%%EndComments
gsave
newpath
  20 20 moveto
  20 220 lineto
  220 220 lineto
  220 20 lineto
closepath
2 setlinewidth
gsave
  .4 setgray fill
grestore
stroke
grestore
\end{filecontents*}
%
\RequirePackage{fix-cm}
%
%\documentclass{svjour3}                     % onecolumn (standard format)
%\documentclass[smallcondensed]{svjour3}     % onecolumn (ditto)
\documentclass[smallextended]{svjour3}       % onecolumn (second format)
%\documentclass[twocolumn]{svjour3}          % twocolumn
%
\smartqed  % flush right qed marks, e.g. at end of proof
%
\usepackage{graphicx}
%
% \usepackage{mathptmx}      % use Times fonts if available on your TeX system
%
% insert here the call for the packages your document requires
%\usepackage{latexsym}
% etc.
%
% please place your own definitions here and don't use \def but
% \newcommand{}{}
%
% Insert the name of "your journal" with
% \journalname{myjournal}
%
\begin{document}

\title{The Furuta Pendulum
%\thanks{Grants or other notes
%about the article that should go on the front page should be
%placed here. General acknowledgments should be placed at the end of the article.}
}
\subtitle{Technical Report}

%\titlerunning{Short form of title}        % if too long for running head

\author{Maximilian Gehrke \and Tabea Wilke \and Yannik Frisch %etc.
}

%\authorrunning{Short form of author list} % if too long for running head

\institute{F. Author \at
              first address \\
              Tel.: +123-45-678910\\
              Fax: +123-45-678910\\
              \email{fauthor@example.com}           %  \\
%             \emph{Present address:} of F. Author  %  if needed
           \and
           S. Author \at
              second address
}

\date{Received: date / Accepted: date}
% The correct dates will be entered by the editor


\maketitle

\begin{abstract}
The Furuta Pendulum is an example of a complex non-linear system and therefore of big interest in control system theory. It consists of one controllable arm rotating in the horizontal plane and one pendulum unconntrollably moving in the vertical plane, which is attached to the end of this arm.\\
The non-linearities result from an interplay between gravitational, Coriolis and centripetal forces. \\
XX We present an overview over it's technical details and proposed algorithms to solve the control problem. XX
\keywords{First keyword \and Second keyword \and More}
% \PACS{PACS code1 \and PACS code2 \and more}
% \subclass{MSC code1 \and MSC code2 \and more}
\end{abstract}

\section{Definitions}
The system consists of an arm with lenght $L_1$ mounted to a DC motor, which is able to apply a torque of $\tau_1$ to it. It has a mass of $m_1$ which is located at $l_1$ alongside the arm. Another arm with length $L_2$ and mass $m_2$ located at $l_2$ along itself is attached to the remaining side of the first arm. Both arms have inertia tensors $J_1$ and $J_2$ respectively and each rotational joint is damped viscously with damping coefficients $b_1$ and $b_2$, where the first coefficient is given by the bearings of the motor and the second one by the coupling between both arms.

\bibliography{furuta}   % name your BibTeX data base


\end{thebibliography}

\end{document}
% end of file template.tex

